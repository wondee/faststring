\chapter{Analyse}

Das folgende Kapitel beschreibt den Analyse Algorithmus, des von mir entworfenen Systems.
Im ersten Abschnitt sollen die verwendeten Datenstrukturen vorgestellt und beschrieben werden.
Der zweite Abschnitt beschreibt schließlich den Algorithmus.

\section{Datenstrukturen}

Für den Algorithmus wurden zwei grundlegende Datenstrukturen entworfen. Der 
\textit{Datenflussgraph} repräsentiert den Datenfluss der Referenzen innerhalb 
einer Methode und wird im ersten Abschnitt vorgestellt. Zu optimierende Referenzen
werden in diesem Graphen mit sogenannten \textit{Labels} versehen. Dieser Datentyp
soll im zweiten Abschnitt beschrieben werden.

\subsection{Datenfluss Graph}\label{ssec:DFG}

Eine auf einem IR basierende Methode wird vom System mittels eines Datenflussgraphen 
repräsentiert. Dieser wird vor der Analyse aus der zu optimierenden
Methode und einer Menge an initialen Referenzen vom \\\texttt{DataFlowGraphBuilder}
erzeugt. Der Graph ist gerichtet und setzt sich aus zwei verschiedenen Knoten zusammen:

\begin{itemize}
	\item \texttt{Reference}: eine \textit{value number} aus dem IR
	\item \texttt{InstructionNode}: eine Instruktion aus dem IR
\end{itemize}

Sei im Folgenden der Datenflussgraph $G = (V, E)$, $R$ die Menge aller \texttt{Reference} 
Knoten und $I$ die Menge aller \texttt{InstructionNode}s. 
\\
Im Graph gilt $\forall (x, y) \in V,  (x \in R \wedge y \in I) \vee (x \in I \wedge y \in X)$.
Eine Kante $i \in I, r \in R, (i, r)$ beschreibt eine \textit{Definition}, die aussagt, 
dass die Referenz $r$ durch die Instruktion $i$ definiert wird. Ein Kante 
$i \in I, r \in R, (r, i)$ ist eine \textit{Benutzung} (im folgenden \textit{Use}
genannt). 

\texttt{Reference} Knoten werden für jede betroffene \textit{value number} erzeugt. Für
die Erzeugung von \texttt{InstructionNode} Objekten steht die 
\texttt{InstructionNodeFactory} zur Verfügung, die für eine gegebene 
\texttt{SSAInstruction} eine entsprechende \\\texttt{InstructionNode} erstellt. Um
für dieselbe \texttt{SSAInstruction} immer dasselbe \texttt{InstructionNode} 
Objekt zu garantieren, verwendet die Factory einen internen Cache, der eine 
Abbildung $SSAInstruction \rightarrow InstructionNode$ verwaltet, und für jede 
\texttt{SSAInstruction} prüft, ob für diese bereits eine \texttt{InstructionNode} 
erstellt worden ist.

Die Erstellung eines \texttt{DataFlowGraph}s beginnt immer mit einer Menge an 
initialen \texttt{Reference} Objekten. Ausgehend von dieser Startmenge werden über 
das \texttt{DefUse}-Objekt des betroffenen IRs die \textit{Definition} und alle Uses, 
dieser Referenzen, in den Datenflussgraphen eingefügt. Algorithmus \ref{alg:dfg} 
beschreibt die Erstellung des Graphen. Die verwendete Queue Implementierung ist eine 
auf einem \texttt{java.util.LinkedHashSet} basierende Eigenentwicklung mit dem Namen
\texttt{de.unifrankfurt.faststring.analysis.util.UniqueQueue}.

\begin{algorithm}[H]
	\caption{Erstellung des Datenflussgraphen}\label{alg:dfg}
	\begin{algorithmic}[1]
		\STATE $q \gets$ \texttt{new Queue($initialReferences$)}
		\STATE $g \gets$ \texttt{new DataFlowGraph()}
		\WHILE {\texttt{not $q$.isEmpty()}}
			\STATE $r \gets q$\texttt{.remove()}
			\IF {\texttt{not $g$.contains()}}

				\STATE $def \gets$ \texttt{$defUse$.getDef($r$)}
				\STATE $uses \gets$ \texttt{$defUse$.getUses($r$)}

				\STATE \texttt{$newInd$.add($def$)}
				\STATE \texttt{$newInd$.add($uses$)}

				\STATE \texttt{$r$.setDef(factory.create($def$))}

				\FOR{\texttt{$ins \in defUse$.getUses($r$)}}
					\STATE \texttt{$r$.addUse(factory.create($ins$))}
				\ENDFOR
				\FOR{$ins \in newIns$}
					\STATE \texttt{$q$.add($ins$.getConnectedRefs())}
				\ENDFOR
				\STATE \texttt{$g$.add($r$))}
			\ENDIF
		\ENDWHILE
		\RETURN $g$
	\end{algorithmic}
\end{algorithm}

Jede \texttt{InstructionNode} besitzt eine \textit{Definition}, die \textit{value number}s der Referenz 
die diese Instruktion erzeugt, und eine Liste von \textit{Use}s, die \textit{value number}s der Referenzen 
die es benutzt. Darüber hinaus besitzt es noch Informationen zu Bytecode Spezifika, die 
im Kapitel \ref{sec:locals} betrachtet werden.

Für verschiedene \texttt{SSAInstruction} Typen existieren entsprechende
\\\texttt{InstructionNode} Subtypen. Allerdings können nicht alle Typen 
einer \\\texttt{SSAInstruction} zugeordnet werden. Im Folgenden sollen die
wichtigsten Knotentypen vorgestellt werden. Es existieren darüber hinaus weitere
Typen von Instruktionen, für die das System zur Zeit keine Unterstützung bietet, 
da es ausschließlich für komplexe Objekte ausgelegt ist.

Einen weiteren Subtyp bildet die \texttt{LabelableNode}. Dieser Typ stellt eine
speziellen Knoten dar, der mit einem Label markiert und damit auch
optimiert werden kann. 

\subsubsection{MethodCallNode}

Eine \texttt{MethodCallNode} repräsentiert einen Methoden Aufruf. Es besitzt, wenn
vorhanden, eine \textit{Definition} (nur vorhanden, bei Zuweisung des Rückgabewert), einen Receiver, 
wenn es keine statische Methode ist, eine Liste an Parametern und die aufgerufene Methode. 
Dieser Typ ist eine \texttt{LabelableNode}.

\subsubsection{ContantNode}

Dieser Knoten Typ stellt eine Konstanten \textit{Definition} dar. Er besitzt ausschließlich 
die \textit{Definition}, welcher Referenz diese Konstante zugewiesen wird. Für den Typ
existiert keine entsprechende \texttt{SSAInstruction}.

\subsubsection{ParameterNode}

Die \texttt{ParameterNode} wird als \textit{Definition} der Parametern der Methode verwendet.
Wird eine Variable innerhalb der Methode als Parameter in der Methoden Signatur
deklariert, wird deren \textit{Definition} als \texttt{ParameterNode} im Datenflussgraphen
repräsentiert. Für diesen Typ existiert keine entsprechende \texttt{SSAInstruction}.

\subsubsection{NewNode}

Dieser Typ entspricht einer \texttt{NEW} Anweisung, die ein neues Objekt eines 
gegebenen Typen erstellt. Es besitzt eine \textit{Definition} und den Typ des instanziierten
Objekts. Die \textit{NewNode} ist eine \texttt{LabelableNode}.

\subsubsection{ReturnNode}

\texttt{ReturnNode} Typen sind \texttt{RETURN} Anweisungen. Sie besitzen (in Java)
ausschließlich eine Referenz als \textit{Use} und zwar diejenige Referenz, die sie aus der Methode 
zurückgeben. Dieser Typ kann keine \textit{Definition} darstellen.

\subsubsection{PhiNode}

Die \texttt{PhiNode} steht für eine $\phi$-Instruktion aus dem IR. Sie verfügt über eine
Referenz als \textit{Definition} und 2 bis $n$ \textit{Use}s. Dieser Typ ist eine \texttt{LabelableNode}.

\subsection{Label}

Das \textit{Label} entspricht einer Markierung, mit der Knoten in einem 
Datenflussgraphen versehen werden können. Dabei steht ein Label (oder 
\texttt{TypeLabel}, wie der Datentyp im System heißt) für einen optimierten Typ.
Die Semantik hinter einem markierten Knoten ist, dass diese Referenz bzw. Instruktion
durch den entsprechenden optimierten Typ ersetzt werden kann.

Es kann nicht für alle \texttt{InstructionNode}s ein Label gesetzt werden. Genauer
gesagt, kann nur die Knotentypen \texttt{MethodCallNode}, \texttt{NewNode} und 
\texttt{PhiNode} ein Label gesetzt werden, da sich nur diese 
Instruktionen in einen optimierte Variante umwandeln lassen.

Das \texttt{TypeLabel} beinhaltet alle Definitionen und Regeln, die für die Verwendung 
eines optimierten Typen existieren. Dazu gehören

\begin{itemize}
	\item der originale, sowie der optimierte Typ
	\item die Methoden, für die der optimierte Typ Optimierungen implementiert
	\item alle Methoden, die darüber hinaus vom Optimierten Typ unterstützt werden
	\item Methoden, die den optimierten Typ als Rückgabewert zurückgeben
	\item zu dem beschriebenen Label kompatible Label
\end{itemize}

Diese Aspekte werden durch die Methoden des Interface abgebildet. 
Im System lassen sich Label Definitionen auf zwei Arten erstellen:

\begin{enumerate} 
	\item Durch das Implementieren des Interfaces \texttt{TypeLabel}
	\item Durch das Erstellen einer \texttt{.type} Datei
\end{enumerate}

Zwar unterstützt das Kommandozeilen Tool zur Zeit nur die zweite Variante,
programmatisch lässt sich allerdings auch die erste umsetzen. Im 
Folgenden sollen die beiden Möglichkeiten zur Definition eines \texttt{TypeLabel}s
betrachtet werden.

\subsubsection{Das TypeLabel Interface} \label{sssec:typeLabel}

Das Interface beinhaltet alle Methoden, die der Analyse- und Transformationsprozess
benötigt. In diesem Kapitel sollen zunächst nur die Methode betrachtet werden, die
für den Analyse Algorithmus verwendet werden, die übrigen werden im Abschnitt 
\ref{ssec:infoLabel} erläutert. 

\begin{description}
	\item[\texttt{canBeUsedAsReceiverFor(MethodReference):boolean}] - Legt fest, ob eine 
	markierte Referenz als Empfänger für den übergebenen Methodenaufruf dienen kann.
	\item[\texttt{canBeUsedAsParamFor(MethodReference,int):boolean}] - Bestimmt, ob eine 
	markierte Referenz als Parameter in dem gegebenen Methodenaufruf an der 
	entsprechenden Stelle (der \texttt{int} Parameter) verwendet werden kann.
	\item[\texttt{canBeDefinedAsResultOf(MethodReference):boolean}] - Sagt aus, ob die 
	gegebene Methode einen optimierten Typ zurückgeben kann. Dies impliziert, dass
	der Methodenaufruf selber auch markiert ist.
	\item[\texttt{findTypeUses(AnalyzedMethod):Set<Reference>}] - Gibt eine Menge an \\\texttt{Reference}
	Objekten zurück, auf denen innerhalb der gegebenen Methode eine der von 
	der Optimierung betroffenen Methode aufgerufen wird. Für diesen Algorithmus 
	existiert bereits eine Implementierung in der Klasse \texttt{BaseTypeLabel}.
	\item[\texttt{compatibleWith(TypeLabel):boolean}] - Gibt an, ob das übergebene Label 
	kompatibel mit diesem Label ist.
\end{description}

All diese Methoden werden von \texttt{InstructionNode} Implementierungen 
verwendet, um Aussagen über die Label Konformität zu treffen. Wie das detailliert geschieht, 
wird im Abschnitt \textit{Algorithmus} beschrieben.

\subsubsection{Das .type Dateiformat}

Da das Implementieren des Interfaces eher komplex ist, wurde für die einfachere 
Definition eines Types ein Dateiformat entwickelt, welches von der Komplexität des
Interfaces abstrahieren soll. In dieser werden nicht die Regeln selbst, sondern 
die Fakten beschrieben, aus denen die Regeln für den Algorithmus hergeleitet werden 
können.

Aus einer Datei im \texttt{type} Format wird mittels eines internen Parsers ein
\texttt{TypeLabelConfig} Objekt erzeugt, welches als \texttt{TypeLabel} Objekt für
den Algorithmus dient.

Für die inhaltliche Struktur der Datei wurde die JSON (JavaScript Object Notation) Syntax 
gewählt, um eine Darstellung anzubieten, die sowohl für Menschen als auch für das 
Programm leicht zu lesen und zu verstehen ist. Die Attribute innerhalb dieser Datei 
werden nachfolgend erläutert. Abbildung \ref{typeFile} zeigt ein Beispiel einer \texttt{type}-Datei. 

\begin{description}
	\item[name] Name des Labels
	\item[originalType] voll qualifizierte Name des zu ersetzenden Typs
	\item[optimizedType] voll qualifizierte Name des zu optimierten Typs
	\item[methodDefs] Liste von Methoden Definitionen. Jede in der Definition verwendeten Methode muss
	in dieser Liste mit einer ID versehen werden. Diese Methode lässt sich über diese
	ID referenzieren und lässt sich nur über diese ID verwenden. Ein Eintrag ist ein Objekt mit
	den Attributen:
	\begin{description}
		\item[id] eindeutige ID für die diese Methode
		\item[desc] Beschreibung dieser Methode. Dies ist ein weiteres Objekt 
		und besteht aus den Attributen:
		\begin{description}
			\item[name] Name der Methode
			\item[signature] Signatur der Methode zusammengesetzt aus der
			Parameterliste und den Rückgabewert. Die Typen müssen dabei in der 
			internen JVM Form angegeben werden. (Beispiel: '\texttt{(I)Ljava/\\lang/String;}''
			, ein Parameter vom Typ \texttt{int} und Rückgabewert vom Typ
			\texttt{java.lang.String})
		\end{description}	
	\end{description}
	\item[effectedMethods] Liste von Methoden IDs. Für diese Methoden existieren 
	optimierte Varianten in dem optimierten Typen.
	\item[supportedMethods] Liste von Methoden IDs. Diese Methoden werden vom
	optimierten Typ zusätzlich unterstützt. Bei den hier angegebenen Methoden handelt 
	es sich nicht um Optimierungen.
	\item[producingMethods] Liste von Methoden IDs. Diese Methoden erzeugen 
	in ihrer optimierten Variante optimierte Typen.
	\item[compatibleLabel] Liste von Strings. Beinhaltet alle Label, die mit diesem Label
	kompatibel sind.
	\item[parameterUsage] Ein Objekt. Dabei ist jeder key die ID einer Methode 
	und der entsprechende value eine Liste von Ganzzahlen. Ein Eintrag bedeutet, 
	dass diese Methode einen optimierten Typ als Parameter mit diesem Index 
	erwartet. 
	\item[optimizedParams] Ein Objekt. Dabei ist jeder key die ID einer Methode 
	und der entsprechende value eine Parameter Liste in interner JVM Notation. 
	Wird bei einer Optimierung einer Methode eine andere Parameter Liste erwartet
	als die ursprüngliche der originalen Methode, so muss die neue Signatur an dieser
	Stelle angegeben werden.
	\item[staticFactory] Ein String. Der Name einer statischen Factory Methode mit 
	einem Übergabeparameter vom Typ des originalen Typs. Diese muss einen entsprechenden
	optimierten Typ zurückgeben.
	\item[toOriginalType] Ein String, Der Name einer Methode ohne Parameter, die
	aus dem optimierten Objekt, ein entsprechendes vom originalen Typ zurückgibt.

\end{description}

\begin{figure}[H]
	\begin{lstlisting}[language=Java]
{
	"name" : "SubstringString",
	"originalType" : "java.lang.String",
	"optimizedType" : 
		"de.unifrankfurt.faststring.core.SubstringString",
	"methodsDefs" : [
		{ 
		  "id" : "substring_i", 	
		  "desc" : { "name" : "substring", 
		  			 "signature" : "(I)Ljava/lang/String;" } 
		},
		{ 
		  "id" : "substring_ii", 	
		  "desc" : { "name" : "substring", 
		  			 "signature" : "(II)Ljava/lang/String;" } 
		},
		{ 
		  "id" : "length", 			
		  "desc" : { "name" : "length", "signature" : "()I" } 
		},
		{ 
		  "id" : "charAt", 
		  "desc" : { "name" : "charAt", "signature" : "(I)C" } },
		{ 
		  "id" : "valueOf", 
		  "desc" : { "name" : "valueOf", 
		  			 "signature" : 
		  			 "(Ljava/lang/Object;)Ljava/lang/String;" }
		}
	],

	"effectedMethods" : ["substring_i", "substring_ii"],
	"supportedMethods" : ["length", "charAt"],
	"producingMethods" : ["substring_i", "substring_ii"],

	"compatibleLabels" : ["StringListBuilder"],

	"staticFactory" : "valueOf",
	"toOriginalMethod" : "toString"
}
	\end{lstlisting} 
	\caption{Beispiel einer \textit{type} Datei, anhand der SubstringString Definition}\label{typeFile}
\end{figure}


\subsection{Konventionen}\label{subs:conventions}

Bei dem Erstellen von optimierten Typen sind einige Konventionen zu befolgen. Das 
System rechnet damit, dass diese Annahmen befolgt werden. 

\subsubsection{Methodennamen}

Wird eine Methode als \textit{effectedMethod} deklariert, wird der optimierte Gegensatz
durch den Namen identifiziert. Wenn z.B. die Methode \texttt{f()} aus dem Typ \texttt{A}
optimiert werden soll, so muss in dem optimierten Typ \texttt{AOpt} eine Methode
\texttt{f()} existieren. Besitzt die originale Methode eine Parameterliste, so muss
diese mit derer der optimierten Methode in Anzahl und Typen übereinstimmen. Eine Ausnahme
bildet dabei die Verwendung von ebenfalls optimierten Parametern. In diesem Fall muss im Feld 
\textit{optimizedParams} in der \textit{type} Datei ein entsprechender Eintrag erfolgen.


\section{Algorithmus}

In diesem Abschnitt soll die Idee hinter dem Analyse Algorithmus sowie dessen
Implementierung vorgestellt werden. Hierzu wird zunächst das Konzept der 'Bubble'
erläutert, um danach auf diesem Konzept aufbauenden Algorithmus zu betrachten.

\subsection{Motivation}

Optimierte Referenzen sind im Falle von String Objekten nicht kompatibel mit der 
originalen Implementierung. So treten Probleme in den folgenden Szenarien auf:

\begin{description}
	\item[Referenz als Methodenparameter] Die Signatur der 
	optimierten Methode, wird nicht verändert, da es dazu führen würde, dass
	Klienten Code, der diese Methode weiterhin mit dem originalen Typ aufruft, 
	nicht mehr kompilieren würde.
	\item[Referenz als Rückgabewert] Ein ähnliches Problem existiert, wenn die 
	Referenz zurückgegeben wird. Die Signatur der Methode definiert den originalen
	Typ als Rückgabetyp und das Zurückgeben eines anderen Typs als der 
	definierte würde zu Laufzeitfehlern führen.
	\item[Methodenaufruf auf Referenz] Wird diese optimierte Referenz als Empfänger 
	von einer Methode verwendet, die nicht zu den optimierten oder unterstützen
	Methoden dieses optimierten Typs gehört, würde es zu Laufzeitfehlern kommen, 
	da diese aufzurufende Methode nicht im optimierten Typ vorhanden ist. 
	\item[Feld Zugriff (\texttt{GETFIELD}, \texttt{PUTFIELD})] Wird die optimierte 
	Referenz in ein oder aus einem Feld innerhalb eines Objektes (oder in ein 
	statisches Feld einer Klasse) gesetzt, stimmt auch in diesem Szenario der Typ
	des optimierten und des originalen Objekts nicht überein.
	\item[Array Zugriff] Bei dem Zugriff auf ein Array, sowohl lesend als auch 
	schreibend, existiert eine Unstimmigkeit mit dem Typ des Arrays.
	\item[Referenz Aufruf Parameter] Wird die Referenz als Parameter für einen
	Methoden Aufruf verwendet, die nach Label Definition nicht mit einem optimierten 
	Typ kompatibel ist, dann erwartet diese Methode den originalen Typ und nicht den optimierten.
\end{description}

Erweitert der optimierte Typ seinen originalen Typ, so wäre durch den 
Polymorphismus der optimierte Typ genauso wie der originale verwendbar. Allerdings 
ist der Typ \texttt{java.lang.String} final, was bedeutet, dass von diesem Typ nicht 
abgeleitet werden kann. Darüber hinaus bieten sich Ableitungen für Optimierungen nicht
an, da das dynamische Dispatchen zusätzlichen Aufwand für die JVM darstellt. Die Begründung liegt 
darin, dass dabei zunächst die Implementierung der Methode zu lokalisieren ist, bevor 
der Methodenaufruf auf nicht finalen Methoden durchgeführt werden kann.

Aus diesem Grund müssen in den Code Konvertierungen zwischen originalem und optimiertem 
Typ eingefügt werden. Eine Konvertierung zum optimierten Typ muss vor dem zu optimierenden
Methodenaufruf erfolgen. Eine Umwandlung vom optimierten Typ muss vor einer 
nicht kompatiblen Benutzung dieser Referenz erfolgen um bei dieser den originalen Typ 
zu verwenden.

\subsection{Die Bubble}

Da Konvertierungen zusätzliche Laufzeit erfordern, muss es das Ziel sein, die Anzahl 
der durchgeführten Konvertierungen zu minimieren. Dies wird erreicht, indem man den
Bereich, in dem ein optimierter Typ statt des originalen innerhalb der Methode 
verwendet wird, maximiert. Dieser Bereich, in dem ein optimierter, statt des originalen 
Typs für eine Referenz verwendet wird, wird im nachfolgend \textit{Bubble} genannt. 

Die Bubble entsteht mittels der Markierung von Knoten im Datenflussgraphen. Es wird 
für jede gegebene Instanz vom Typ \texttt{TypeLabel} eine Bubble auf dem Graphen 
erzeugt. Dabei kann ein Knoten mit maximal einem Label markiert sein, daher kann
ein Knoten immer nur Teil einer Bubble sein.

Ziel des Algorithmus ist es, diese Bubble ausgehend von den zu optimierenden Methoden 
so weit wie möglich zu erweitern. Als Anfangszustand werden alle zu optimierenden 
Methodenaufrufe mit dem zu verarbeitenden Label markiert, die alle für sich einzelne
Bubbles darstellen.

\subsection{Umsetzung des Algorithmus}\label{ssec:umAlg}

Als Eingabe für den Algorithmus dient ein Objekt vom Typ \texttt{DataFlowGraph}, der 
den bereits beschriebenen Datenflussgraphen darstellt. Zum Erstellen des Graphen werden
wie in \ref{ssec:DFG} beschrieben zunächst initiale Referenzen benötigt. Diese Referenzen
werden über die Methode \texttt{Set<Reference>} \texttt{findTypeUses\\(AnalyzedMethod)} ermittelt, 
welche sich in der abstrakten Klasse \\\texttt{BaseTypeLabel} befindet. Diese Methode ist auch Teil des 
\texttt{TypeLabel} Interfaces. Die betreffenden Methoden werden mittels der in der Label 
Definition definierten \textit{effectedMethods} identifiziert, also derjenigen Methoden, für die dieses 
Label eine Optimierung darstellt. Die \texttt{findTypeUses} Methode erstellt für jede 
Referenz, auf der in der gegebenen Methode eine der \textit{effectedMethods} aufgerufen 
wird, ein \texttt{Reference} Objekt. Dabei wird für jede \textit{value number} genau ein Referenz 
Objekt erzeugt. Die erstellten Referenzen werden dem verwendeten Label markiert.

Die Implementierung des Algorithmus befindet sich als statische Methode in der Klasse 
\texttt{LabelAnalyzer}. Ausgehend von den initial markierten Knoten, werden jeweils die 
\textit{Definition} und \textit{Use}s einer Referenz betrachtet, ob diese mit dem Label der Referenz 
markiert werden können. Der Algorithmus \ref{alg:analyze} beschreibt die Implementierung 
der Vererbung des Labels. Als Queue findet die 
\texttt{de.unifrankfurt.faststring.analysis.util.UniqueQueue} Verwendung.

\begin{algorithm}[H]
	\caption{Vererbung des Labels}\label{alg:analyze}
	\begin{algorithmic}[1]
		\STATE $q \gets$ \texttt{new Queue($initialReferences$)}
		\STATE $g \gets$ der übergebene Datenflussgraph
		\WHILE {\texttt{not $q$.isEmpty()}}
			\STATE $r \gets q$\texttt{.remove()}
			\IF {\texttt{$r$.label} is not \texttt{null}}
				\STATE $def \gets$ \texttt{$r$.getDef()}
				\IF {\texttt{$def$.canProduce($r$.label)}}
					\STATE \texttt{labelConnectedRefs($def$)}
				\ENDIF
				\FOR {$use$ \bf{in} \texttt{$r$.getUses()}}
					\IF {\texttt{$use$.canUseAt($r$.label, $useIndex$)}}
						\STATE \texttt{labelConnectedRefs($use$)}
					\ENDIF
				\ENDFOR
			\ENDIF
		\ENDWHILE
		\RETURN $g$
	\end{algorithmic}
\end{algorithm}

Eine wichtige Rolle während des Vererbungsprozesses spielen dabei die Methoden 
\texttt{boolean Labelable.canProduce(TypeLabel)} und
\texttt{boolean Labelable\\.canUseAt(TypeLabel, int)}. Diese beiden Methoden bestimmen 
für eine Referenz, ob ein Label auf eine \textit{Definition} oder ein \textit{Use} vererbt werden kann.

\texttt{canProduce} sagt aus, ob eine Instruktion eine optimierte Referenz definieren kann. 
Die Implementierungen der \texttt{canProduce} Methode sehen für die einzelnen \texttt{Labelable}
Subtypen wie folgt aus:

\begin{description}
	\item[NewNode] gibt immer \texttt{true} zurück.
	\item[ConditionalBranchNode] gibt immer \texttt{false} zurück.
	\item[MethodCallNode] delegiert den Aufruf an die Methode \texttt{canBeDefinedAsResultOf} der
	übergebenen Label Instanz.
\end{description}

Die Methode \texttt{canUseAt} betrifft dagegen Verwendungen von Referenzen. Die Methode beantwortet 
die Frage, ob diese Instruktion eine optimierte Referenz verwenden kann. Der übergebene Integer Wert 
stellt dabei den Usage Index dar, an dem diese Referenz in dieser Instruktion verwendet wird. 
Die Implementierung dieser Methode sieht für die einzelnen \texttt{Labelable}
Subtypen wie folgt aus:

\begin{description}
	\item[NewNode] gibt immer \texttt{false} zurück.
	\item[ConditionalBranchNode] gibt immer \texttt{true} zurück.
	\item[MethodCallNode] Wenn die Methode nicht statisch und der Index = 0 ist, wird an die Methode
	\texttt{canBeUsedAsReceiverFor} delegiert, andernfalls an die Methode \texttt{canBeUsedAsParamFor}.
\end{description}

Ist für eine Instruktion die Prüfung für die entsprechende Methode positiv, wird diese 
Instruktion an die Hilfsfunktion \texttt{labelConnectedRefs} weitergeleitet.

\begin{algorithm}[H]
	\caption{labelConnectedRefs}\label{alg:labelConnRefs}
	\begin{algorithmic}[1]
		\STATE \texttt{$node$.setLabel($label$)}
		\FOR{$ref$ \bf{in} $node$\texttt{.getLabelableRefs()}}
			\IF{\texttt{$ref$.getLabel()} is not \texttt{null}}
				\STATE \texttt{$ref$.setLabel($label$)}
				\STATE \texttt{$q$.add($ref$)}
			\ENDIF 
		\ENDFOR
	\end{algorithmic}
\end{algorithm}

Wobei sowohl die Queue $q$, als auch das betreffende TypeLabel $label$ aus dem umgebenen Kontext
im Algorithmus \ref{alg:analyze} in dieser Funktion zur Verfügung stehen. 

\subsection{Umgang mit mehreren Labels}

Es ist möglich für die Analyse mehrere Labels für die Analyse zu verwenden. Diese lassen
sich dem \texttt{MethodAnalyzer} als \texttt{Collection} im Konstruktor zusammen 
mit der zu analysierenden Methode übergeben. Vor der Analyse werden zuerst alle Referenzen 
identifiziert, von denen eine zu optimierende Methode verwendet wird. Diese Referenzen werden 
zu einer Menge an Referenzen vereint. Die auf diese Weise erzeugten Referenzen besitzen, 
wie in \ref{ssec:umAlg} beschrieben, jeweils das Label, für das die Referenz erzeugt wurde.

Stellt man der Analyse mehrere Labels zur Verfügung, die allerdings Optimierungen für 
dieselbe Methoden desselben Typs beschreiben, ergibt sich ein Problem. Da jede 
Referenz immer nur mit einem Label markiert werden kann, werden diejenigen Referenzen, 
welche dieselbe zu optimierende Methode verwenden, mit demjenigen Label markiert, welches 
zuerst verarbeitet wird. Das wird immer das Label sein, welches einen geringeren Index in 
der Liste von übergebenen Labels besitzt. 

Stellt man der Methode nun zwei unterschiedlichen Labels zur Verfügung, die denselben 
originalen Typen verwenden, ergibt sich ein anderes Problem, und zwar wenn 
innerhalb der analysierten Methode auf einer Referenz diesen Typs sowohl die eine 
als auch die andere zu optimierende Methode aufgerufen wird. Da jeder Referenz in 
einer Methode maximal ein Label zugewiesen wird, kann die betroffene Referenz, welche 
beide Methoden der beiden TypeLabels verwendet, nur mit einem der TypeLabels markiert werden. 
In diesem Szenario wird ebenfalls dasjenige Label für die Markierung verwendet, welches den niedrigeren
Index innerhalb der Liste besitzt.

Nachdem der Datenflussgraph initial erzeugt wurde, wird der Algorithmus wie in \ref{ssec:umAlg} 
beschrieben ausgeführt. Da Labels nur auf Knoten vererbt werden, gilt es dass der entsprechende
Knoten dasjenige Label erhält, welches zuerst auf diesen Knoten vererbt wird.  

\subsection{Phi-Knoten}

$\phi$-Instruktionen innerhalb des Datenflussgraphen erfordern aufgrund ihrer Eigenheiten
eine besondere Behandlung. Da sie keine Instruktionen im eigentlichen Sinne darstellen, sondern
nur ein Zeiger Alias für andere Referenzen sind, lassen sich diese Knoten nicht
optimieren. Obwohl sie selber nicht optimiert werden können, lässt sich doch über diese 
Knoten hinweg ein Label vererben. Allerdings sollte unterschieden werden zwischen denjenigen 
Referenzen innerhalb Instruktion, für die eine Markierung tatsächlich sinnvoll und für 
welche eher weniger sinnvoll ist. Es sollten diejenigen Referenzen markiert werden, für die 
auch Optimierungspotenzial besteht. Wohingegen die Referenzen, für die keinerlei 
Optimierung notwendig ist, keine Markierung erhalten sollten. 

Wenn das System beim Vererben der Markierung einen $\phi$-Knoten als Instruktion verarbeitet,
wird dieser auf den $\phi$ Stapel gelegt und alle verbundenen Referenzen für die
weitere Verarbeitung vorgemerkt. Nachdem alle Referenzen verarbeitet worden sind, wird dieser
Stapel durchlaufen. Für jede \texttt{PhiInstructionNode} in diesem Stapel wird entschieden,
ob und mit welchem Label diese Instruktion markiert wird. Diese Entscheidung wird über
die Häufigkeit einer Label-Markierung in der Menge aller verbundenen Referenzen 
getroffen. 

Seien $L = \{l_1,...,l_n\}$ alle Labels der aktuellen Analyse, inklusive des Labels 
für keine Optimierung, in ihrer natürlichen Reihenfolge und die Funktion 
$anzahl: L \rightarrow \mathbb{N}$ die Anzahl der Markierungen aller verbunden Referenzen. 
Dann ist $m = max(\{anzahl(l_i) | l_i \in L\})$ die Markierung für den verarbeiteten $\phi$-Knoten, 
wenn $m$ eindeutig ist. Ist das Maximum nicht eindeutig, so wird aus der Menge der 
Labels mit der größten Häufigkeit, dasjenige gewählt, welches den geringsten Index besitzt. 