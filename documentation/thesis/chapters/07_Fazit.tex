\chapter{Fazit}

Dieses Kapitel fasst die Ergebnisse dieser Arbeit zusammen, zieht ein Fazit über
die Ergebnisse der Auswertungen und gibt Ausblicke auf zukünftige Arbeit an dem System, 
um die hier gezeigten Ergebnisse zu verbessern. 

In dieser Arbeit wurde ein System erstellt, das anhand statischer Codeanalyse und
automatischer Transformation Optimierungen anwenden kann. Es wurde die Analyse des
Systems vorgestellt, in welchem Konzept des \textit{TypeLabel}s erläutert. Das Ersetzen der 
originalen Datentypen durch optimierte Alternativen, sowie das Hinzufügen von Code
zum Konvertieren der beiden Datentypen untereinander, wurde betrachtet und beschrieben.
Schließlich wurde das System und die entwickelten String Optimierungen anhand eines 
einfachen Beispiels und eines komplexen Software Systems (der \textit{Xalan} Bibliothek)
getestet, indem die Laufzeit der transformierten Programme gemessen wurde. 

Die Auswertung zeigt, dass die Optimierung durch automatischer Transformation von Programmen
schwierig ist. Zwar lässt sich mit dem \textit{ExampleParser} Benchmark zeigen,
dass Optimierungen generell mit dem System möglich sind, doch ergeben sich Probleme
bei der Anwendung in komplexeren Programmen. Das belegen die \textit{Xalan} Benchmarks. 
Es zeigt sich, dass in komplexen Programmen die Kontexte, in denen Referenzen verwendet
werden, die Optimierungsmöglichkeiten des Systems stark beeinflussen. 

Wie in den Benchmarks \textit{instantiateURI} und \textit{compileXPath} zu sehen ist,
ergeben sich Schwierigkeiten bei der Übergabe von optimierten Referenzen als Parameter. 
Da die hier erarbeitete Lösung ausschließlich intraprozeduraler
Natur ist, führen derartige Benutzungen von Referenzen zwangsläufig zu zusätzlichen 
Konvertierungen. Der \textit{compileXPath} Test zeigt darüber hinaus, dass zu
optimierende Referenzen innerhalb des eingebetteten Methodenaufrufs ebenfalls von 
Optimierungen betroffen sind. Die Implementierung einer interprozeduralen Expansion
der \textit{Bubble} weist daher zusätzliches Potenzial auf, Optimierungen effizienter zu gestalten.

Die Implementierung der interprozeduralen Analyse und Transformation birgt zusätzliche
Herausforderungen. So ist zunächst in der Analyse festzustellen, welche Implementierung einer 
Methode aufgerufen wird. Durch den Polymorphismus, die Nutzung der \textit{Reflections} API und
dynamisches Klassenladen zur Laufzeit, ist dies nicht über statische Code Analyse alleine zu 
lösen, sondern erfordert eine Analyse zur Laufzeit des Programms. Mittels dieses Ansatzes wäre 
es möglich die Ziele von abstrakten Methodenaufrufen zu identifizieren und diese für die 
Analyse zu verwenden. 

Ein weiteres Problem ergibt sich bei der Transformation in einem interprozeduralen Kontext. Bei Methoden, 
deren Parameter in optimierter Form übergeben werden, muss die Signatur angepasst werden, um
kompatibel mit dem optimierten Typ zu sein. Handelt es sich bei der Methode allerdings um einen
Teil der öffentlichen API, so darf diese Signatur nicht ersetzt werden. Damit würde ein Klienten Programm,
das diese Methode verwendet und selber nicht optimiert wurde, zu Laufzeitfehler führen.  

Das in dieser Arbeit erstellte System bietet durch die \textit{TypeLabel} Schnittstelle, die 
Möglichkeit verschiedenste Optimierungen von Typen zu verarbeiten. Daher ist
es nicht auf die Optimierung von Strings beschränkt. Da diese Arbeit 
ausschließlich auf die von Optimierungen für die String API eingeht, wurden
auch keine anderen Optionen in Erwägung gezogen. Dabei schließt das System die Anwendung 
alternativer Optimierungen nicht aus, sondern ist auf alle Typen des Java Typ Systems anwendbar. 

Als Kandidaten wäre die Dezimalzahl Repräsentation \texttt{java.math.BigDecimal} zu betrachten,
die zwar exakte Darstellung von Nachkommastellen, aber nicht die Performanz eines \texttt{float} oder
\texttt{double} Wertes, bietet. Es wäre eine Repräsentation des \texttt{BigDecimal} Wertes denkbar,
die für Darstellung einer Ganzzahl optimiert ist. Dabei ergibt sich die Schwierigkeit, dass mit Hilfe von
statischer Code Analyse nicht identifizierbar ist, ob eine Variable eine Ganzzahl oder eine
Dezimalzahl darstellt. Für eine solche Analyse müsste ebenfalls die dynamische Code Analyse verwendet werden, 
die feststellt, wie eine Referenz während der Laufzeit verwendet wird und welche Werte sie annimmt.

