\documentclass[12pt]{report}


\usepackage[T1]{fontenc}
\usepackage[utf8]{inputenc} 
\usepackage[ngerman]{babel}

\usepackage{setspace}
\usepackage{xcolor}
\usepackage{listings}
\usepackage{algorithm}
\usepackage{algorithmic}
\usepackage{float}
\usepackage{amsfonts}
\usepackage{graphicx}
\usepackage{titlesec}
\usepackage[linktoc=all]{hyperref}
\hypersetup{
    colorlinks,
    citecolor=black,
    filecolor=black,
    linkcolor=black,
    urlcolor=black
}


\lstset{basicstyle=\ttfamily,breaklines=true}

\definecolor{javared}{rgb}{0.6,0,0} % for strings
\definecolor{javagreen}{rgb}{0.25,0.5,0.35} % comments
\definecolor{javapurple}{rgb}{0.5,0,0.35} % keywords
\definecolor{javadocblue}{rgb}{0.25,0.35,0.75} % javadoc
 
\lstset{language=Java,
basicstyle=\footnotesize\ttfamily,
keywordstyle=\color{javapurple}\bfseries,
stringstyle=\color{javared},
commentstyle=\color{javagreen},
morecomment=[s][\color{javadocblue}]{/**}{*/},
%numbers=left,
%numberstyle=\tiny\color{black},
%stepnumber=1,
%numbersep=10pt,
tabsize=4,
showspaces=false,
showstringspaces=false}

\lstset{literate=%
    {Ö}{{\"O}}1
    {Ä}{{\"A}}1
    {Ü}{{\"U}}1
    {ß}{{\ss}}1
    {ü}{{\"u}}1
    {ä}{{\"a}}1
    {ö}{{\"o}}1
    {~}{{\textasciitilde}}1
}


\lstdefinelanguage{Bytecode}{
  keywords={GETSTATIC, INVOKEVIRTUAL, LDC, RETURN, ICONST, ILOAD, IADD, ISTORE},
  keywordstyle=\color{black}\bfseries,
  identifierstyle=\color{black},
  sensitive=false,
  comment=[l]{//},
  morecomment=[s]{/*}{*/},
  commentstyle=\color{purple}\ttfamily,
  stringstyle=\color{red}\ttfamily,
  morestring=[b]',
  morestring=[b]",
  basicstyle=\footnotesize\ttfamily
}

\setcounter{secnumdepth}{2}

\titleformat{\paragraph}{\normalfont\normalsize\bfseries}{\theparagraph}{1em}{}
\titlespacing*{\paragraph}{0pt}{3.25ex plus 1ex minus .2ex}{1.5ex plus .2ex}

\title{Optimierung mittels der Auswahl von String Repräsentationen in Java Bytecode}
\author{Markus Wondrak\\Goethe Universität Frankfurt am Main\\http://www.sepl.informatik.uni-frankfurt.de}
\date{\today}

\begin{document}
\maketitle
\pagenumbering{roman}
\tableofcontents
\listoffigures
\listofalgorithms

\begin{abstract}
Es geht um blibla blubb
\end{abstract}
\pagenumbering{arabic}


\chapter{Einleitung}

In dieser Arbeit soll untersucht werden, ob es möglich ist...

\chapter{Werkzeuge}

In den folgenden Abschnitten sollen die verwendeten Werkzeuge kurz vorgestellt werden. Dabei handelt es 
sich sowohl um den Java Bytecode, als auch um die Software Bibliothek \textit{WALA}, auf deren API das 
in dieser Arbeit entwickelte System basiert. 

\section{Java Bytecode}

Die Plattformunabhängigkeit, die in Java geschriebenen Programmen zugesprochen wird, 
ist vor allem mit der Rolle der Java Virtual Machine zu erklären. Java Programme werden 
in einen Zwischencode, den Java Bytecode, übersetzt, welcher von der System spezifischen 
JVM ausgeführt wird. Dabei ist Programmiersprache Java nicht die einzige in Bytecode 
übersetzbare Sprache. Es existieren neben den bekanntesten Scala, Jython, Groovy,
JavaScript noch viele weitere. Einmal in Bytecode übersetzt können, in diesen Sprachen 
geschriebene, Programme auf jeder, der Java Spezifikation entsprechenden, JVM ausgeführt 
werden. 

Bytecode ist eine Sammlung von Instruktionen welche durch \textit{opcodes} von 2 Byte Länge definiert 
werden. Zusätzlich können noch 1 bis $n$ Parameter verwendet werden. Die Sprache ist Stack-orientiert, das 
bedeutet, dass von Operationen verwendete Parameter über einen internen Stack übergeben werden. Als Beispiel 
dient der folgende Bytecode:

\begin{figure}[H]
	\begin{lstlisting}[language=Bytecode]
		ICONST 5 	// legt den konstanten int Wert 5 auf den Stack 
		ILOAD 1		// läd die lokale integer Variable 1 und legt sie auf den Stack
		IADD 		// addiert die ersten beiden Werte auf dem Stack und legt das Ergebnis auf den Stack
		ISTORE 2	// speichert den Wert auf dem Stack in der Variable 2
	\end{lstlisting} 
	\caption{Java Bytecode Beispiel}
\end{figure}

Dabei existiert der Stack nur als Abstraktion für den eigentlichen Prozessor im Zielsystem. Wie die 
jeweilige JVM den Stack in der Ziel Plattform umsetzt ist nicht definiert. Die Instruktionen lassen sich in 
folgende Kategorien einordnen:

\begin{itemize}
	\item Laden und Speichern von lokalen Variablen (\texttt{ILOAD}, \texttt{ISTORE})
	\item Arithmetische und logische ausdrücke (\texttt{IADD})
	\item Object Erzeugung bzw. Manipulation (\texttt{NEW}, \texttt{PUTFIELD})
	\item Stack Verwaltung (\texttt{POP}, \texttt{PUSH}) 
	\item Kontrollstruktur (\texttt{IFEQ}, \texttt{GOTO})
	\item Methoden Aufrufe (\texttt{INVOKEVIRTUAL}, \texttt{INVOKESTATIC})
\end{itemize}  


\section{WALA}

Bei \textit{WALA} handelt es sich um die "T.J. Watson Library for Analysis". Eine ehemals von IBM 
entwickelte Bibliothek für die statische Codeanalyse von Java- und JavaScript Programmen. Das Framework 
übernimmt dabei das Einlesen von \textit{class} Dateien und stellt eine Repräsentation, die sogenannte 
\textit{Intermediate Representation}, des Bytecodes zur Verfügung. Diese IR stellt die zentrale 
Datenstruktur dar und soll in diesem Abschnitt detailliert beschrieben werden.

Für die Manipulation des Bytecodes existiert innerhalb des Frameworks ein Unterprojekt, das diese Aufgabe 
übernimmt: Shrike. Im Zweiten Abschnitt soll diese API kurz vorgestellt werden.  

\subsection{IR}

Die \textit{Intermediate Representation} (IR) ist eine Abstraktion zum Stack basierten Bytecode. Ein IR ist
in Single Static Assignment Form, welche sich dadurch auszeichnet, dass jeder Variablen immer genau 
\textbf{einmal} ein Wert zugewiesen wird. Zusätzlich besteht die IR aus dem Kontrollflussgraphen der 
Methode, welcher wiederum aus Basic Blocks zusammengesetzt ist. Ein Basic Block ist eine Zusammenfassung 
von aufeinander Folgende Instruktionen, welche in jedem Fall nach einander ausgeführt werden.

Die Variablen innerhalb des IRs nennt WALA \textit{value numbers}. Diese beziehen sich immer auf eine 
Referenz, allerdings kann sich eine Referenz sich auf mehrere tatsächliche value numbers in der IR beziehen.
Dies folgt aus der SSA Form, wird eine Variable im Bytecode zweimal ein Wert zugewiesen, wird diese 
doppelte Zuweisung in der SSA Form durch das Einführen einer neuen value number entfernt. Die Operationen
werden auch nur mit Bezug auf die value numbers beschrieben.   
 
Da die Zwischendarstellung vom Stack abstrahieren soll, werden auch alle Operationen, die den Stack betreffen
(wie z.B. \texttt{LOAD}, \texttt{STORE}, \texttt{PUSH} oder \texttt{POP}) nicht nicht mit in diese 
Repräsentation übernommen. Dabei werden die Bytecode Indices der übrig gebliebenen Instruktionen 
berücksichtigt und alle anderen Stellen in dem beinhaltenden Array mit \texttt{null} Werten aufgefüllt.
Instruktionen werden von Objekten vom Typ \texttt{SSAInstruction} und dessen Untertypen dargestellt.

Die Verwaltung der value numbers wird von einem Typ namens \texttt{SymbolTable} übernommen. Es kommt bei der
IR Erstellung zum Einsatz, wenn bei der Simulation des Bytecodes neue Variablen verwendet werden, um neue 
value numbers zu erzeugen.

Aufgrund der SSA Form der IR lässt sich für jede value number genau eine Definition bestimmen. Zu diesem 
Zweck bietet WALA den Typ \texttt{DefUse} an, welcher für jedes IR-Objekt erstellt werden kann. Er ermöglicht
einen einfachen Zugriff aus die Instruktionen, die eine value number definiert (\textit{def}; z.B. als 
Rückgabe aus einem Methodenaufruf), und eine Menge an Instruktionen, die die entsprechende value number 
verwenden (\textit{use}; z.B. als Rückgabewert in einem \texttt{RETURN} Statement).

Besitzt ein Block im Kontrollflussgraphen mehrere eingehende Kanten und werden aus diesen Vorgänger Blöcken
Variablen mitgebracht die synonym in diesem Block verwendet werden, werden in SSA-Form sogenannte $\phi$ 
Funktionen verwendet. Eine Instruktion der Form $v_3 = \phi(v_1, v_2)$ sagt aus, dass im Folgenden 
die Referenz $v_3$ sowohl $v_1$, als auch $v_2$ sein kann. Da die statische Code Analyse nicht feststellen
kann von welchem Block aus dieser Block betreten wurde, werden diese $\phi$-Funktionen verwendet, um
die Zusammenführungen von mehreren Variablen aus Vorgängerblöcken darzustellen.

Das IR und das dazugehörige DefUse Objekt werden in dem System internen Datentyp \texttt{AnalyzedMethod} 
zusammengefasst.

\subsubsection{Anpassungen}

In WALA werden beim Erstellen des IR für alle Konstanten mit demselben Wert dieselbe value numbers erzeugt. 
Da für die \textit{Analyse} verschiedenen Referenzen getrennt getrennt untersucht werden mussten, wurde 
für die Erzeugung einer value number für eine Konstante der eingebaute caching Mechanismus umgangen. 

Darüber hinaus war für die \textit{Transformation} die Information nötig, an welcher Stelle im Bytecode eine
entsprechende Konstante erzeugt wird (z.B. mittels \texttt{LCD}). Um dies zu Erreichen wurde dieser Bytecode 
Index während dem Durchlaufen der Instruktionen innerhalb der \texttt{SymbolTable} gespeichert, sodass er 
beim Klienten des IRs zur Verfügung steht.

Da diese Änderungen nicht in den Haupt Branch von WALA eingepflegt werden durften, benötigt das System den 
Fork des WALA Projektes \footnote{Dieser ist unter \texttt{http://github.com/wondee/WALA} zu finden.}.


\subsection{Shrike}

Shrike ist ein Unterprojekt innerhalb des WALA Frameworks. Shrike übernimmt dabei das Lesen und das 
Schreiben von Bytecode aus bzw. in class Dateien. Dabei wird es zum einen beim Erstellen eines IR aus einer 
Methode verwendet, zum Anderen bietet es eine "Patch-based" API an um den Bytecode einer eingelesenen 
Methode zu verändern. Dies geschieht über das Einfügen von \texttt{Patch}es, welche über einen 
entsprechenden \texttt{MethodEditor} überall im Bytecode einer Methode eingefügt werden oder auch 
ursprüngliche Instruktionen komplett ersetzen. Zusätzlich enthält es einen \texttt{Verifier}, der erzeugten 
Bytecode überprüft, so dass ungültige Stack Zustände oder Typfehler noch während der Manipulation erkannt 
werden können. 

In dem von mir entwickelten System werden alle Bytecode Manipulationen mit Hilfe von Shrike umgesetzt. 

\chapter{Optimierte Stringtypen}\label{stringLabels}

Im Rahmen dieser Arbeit wurden zwei alternative String Typen die Optimierungen der ursprünglichen
\texttt{java.lang.String} API darstellen. Diese Typen umgehen das Design der String Repräsentationen
in Java, die \textit{nicht veränderbare} Objekte darstellen. Durch diesen Umstand führen alle
Manipulationsoperationen auf String Typen dazu, dass die Daten auf denen diese Objekte basieren
(ein \texttt{char} Array, dass die einzelnen Zeichen der Zeichenkette hält) kopiert werden müssen.
Es sind über die hier vorgestellten Typen durchaus weitere Optimierungen für den String Typ denkbar, 
doch wurden im Rahmen dieser Arbeit nur diese beiden Typen betrachtet. Diese Typen befinden sich
im Maven Artefakt \textit{faststring-core}.

\section{SubstringString} 

Der Typ \texttt{de.unifrankfurt.faststring.core.SubstringString} dient als Optimierung für
die Methode \texttt{java.lang.String.substring(..)}. Dabei dient das \texttt{char} Array als
Daten für die repräsentierte Zeichenkette. 

\section{StringListBuilder} 
\chapter{Analyse}

Das Folgende Kapitel beschreibt den Analyse Algorithmus, des von mir entworfenen Systems.
Im ersten Abschnitt sollen die verwendeten Datenstrukturen vorgestellt und beschrieben werden.
Der zweite Abschnitt beschreibt den eigentlichen Algorithmus.

\section{Datenstrukturen}

Für den Algorithmus wurden zwei grundlegende Datenstrukturen entworfen. Der 
\textit{Datenflussgraph} repräsentiert den Datenfluss der Referenzen innerhalb 
einer Methode und wird im ersten Abschnitt vorgestellt. Zu optimierende Referenzen
werden in diesem Graphen mit sogenannten \textit{Labels} versehen. Dieser Datentyp
soll im zweiten Abschnitt beschrieben werden.

\subsection{Datenfluss Graph}

Eine auf einem IR basierende Methode wird vom System mittels eines Datenflussgraphen 
repräsentiert. Dieser wird vor der eigentlichen Analyse aus einer gegeben
Methode und einer Menge an initialen Referenzen vom \texttt{DataFlowGraphBuilder}
erzeugt. Der Graph ist gerichtet und setzt sich aus zwei verschiedenen Knoten zusammen:

\begin{itemize}
	\item \texttt{Reference}: eine value number aus dem IR
	\item \texttt{InstructionNode}: eine Instruktion aus dem IR
\end{itemize}

Sei im Folgenden der Datenflussgraph $G = (V, E)$, $R$ die Menge aller \texttt{Reference} 
Knoten und $I$ die Menge aller \texttt{InstructionNode}s. 
\\
Im Graph gilt $\forall (x, y) \in V,  (x \in R \wedge y \in I) \vee (x \in I \wedge y \in X)$.
Eine Kante $i \in I, r \in R, (i, r)$ beschreibt eine \textit{Definition}, die aussagt, 
dass die Referenz $r$ durch die Instruktion $i$ definiert wird. Ein Kante 
$i \in I, r \in R, (r, i)$ ist eine \textit{Benutzung} (im folgenden \textit{Use}
genannt). 

\texttt{Reference} Knoten werden für jede betroffene value number erzeugt. Für
die Erstellung von \texttt{InstructionNode} Objekten steht die 
\texttt{InstructionNodeFactory} zur Verfügung, die für eine gegebene 
\texttt{SSAInstruction} eine entsprechende \texttt{InstructionNode} erstellt. Um
für dieselbe \texttt{SSAInstruction} immer dasselbe \texttt{InstructionNode} 
Objekt zu garantieren verwendet die Factory einen internen Cache, der eine 
Abbildung $SSAInstruction \rightarrow InstructionNode$ verwaltet und für jede 
\texttt{SSAInstruction} prüft ob für diese bereits eine \texttt{InstructionNode} 
erstellt wurde.

Die Erstellung eines \texttt{DataFlowGraph}s beginnt immer mit einer Menge an 
initialen \texttt{Reference} Objekten. Ausgehend von dieser Startmenge werden über 
das \texttt{DefUse}-Objekt des betroffenen IRs die Definition und alle Uses in den 
Datenflussgraphen eingefügt. Algorithmus \ref{alg:dfg} beschreibt die Erstellung des 
Graphen.

\begin{algorithm}[H]
	\caption{Erstellung des Datenflussgraphen}\label{alg:dfg}
	\begin{algorithmic}[1]
		\STATE $q \gets$ \texttt{new Queue($initialReferences$)}
		\STATE $g \gets$ \texttt{new DataFlowGraph()}
		\WHILE {\texttt{not $q$.isEmpty()}}
			\STATE $r \gets q$\texttt{.remove()}
			\IF {\texttt{not $g$.contains()}}

				\STATE $def \gets$ \texttt{$defUse$.getDef($r$)}
				\STATE $uses \gets$ \texttt{$defUse$.getUses($r$)}

				\STATE \texttt{$newInd$.add($def$)}
				\STATE \texttt{$newInd$.add($uses$)}

				\STATE \texttt{$r$.setDef(factory.create($def$))}

				\FOR{\texttt{$ins \in defUse$.getUses($r$)}}
					\STATE \texttt{$r$.addUse(factory.create($ins$))}
				\ENDFOR
				\FOR{$ins \in newIns$}
					\STATE \texttt{$q$.add($ins$.getConnectedRefs())}
				\ENDFOR
				\STATE \texttt{$g$.add($r$))}
			\ENDIF
		\ENDWHILE
		\RETURN $g$
	\end{algorithmic}
\end{algorithm}

Jede \texttt{InstructionNode} besitzt eine Definition, die Nummer der Referenz 
die diese Instruktion erzeugt und eine Liste von Uses, die Nummern der Referenzen 
die es benutzt. Darüber hinaus noch Informationen zu Bytecode Spezifika, die 
im Kapitel \ref{sec:locals} betrachtet werden.

Für verschiedene \texttt{SSAInstruction} Typen existieren entsprechende 
\texttt{InstructionNode} Subtypen. Allerdings gibt es auch Typen die nicht
einer \texttt{SSAInstruction} zugeordnet werden können. Im Folgenden sollen die
wichtigsten Knotentypen vorgestellt werden. Es existieren darüber hinaus weitere
für die das System zur Zeit keine Unterstützung bietet, da es ausschließlich für 
String Typen und komplexe Objekte ausgelegt ist.

\subsubsection{MethodCallNode}

Eine \texttt{MethodCallNode} repräsentiert einen Methoden Aufruf. Es besitzt, wenn
vorhanden, eine Definition, welche den Rückgabewert repräsentiert, einen Receiver, 
wenn es keine statische Methode ist und eine Liste an Parametern. Zusätzlich die 
aufgerufene Methode. 

\subsubsection{ContantNode}

Dieser Knoten Typ stellt eine Konstanten Definition dar. Er besitzt ausschließlich 
die Definition, welcher Referenz diese Konstante zugewiesen wird. Für diesen Typ
existiert keine entsprechende \texttt{SSAInstruction}.

\subsubsection{ParameterNode}

Die \texttt{ParameterNode} stellt eine Definition eines Parameters der Methode dar.
Wird eine Variable innerhalb der Methode als Parameter in der Methoden Signatur
deklariert, wird deren Definition als \texttt{ParameterNode} im Datenflussgraphen
repräsentiert. Für diesen Typ existiert keine entsprechende \texttt{SSAInstruction}.

\subsubsection{NewNode}

Dieser Typ entspricht einer \texttt{NEW} Anweisung, die ein neues Objekt eines 
gegebenen Typen erstellt. Es besitzt eine Definition und den Typ des instanziierten
Objekts. 

\subsubsection{ReturnNode}

\texttt{ReturnNode} Typen sind \texttt{RETURN} Anweisungen. Die besitzen 
ausschließlich eine Referenz als Use. Diejenige Referenz, die sie aus der Methode 
zurückgeben. Dieser Typ kann keine Definition darstellen.

\subsubsection{PhiNode}

Die \texttt{PhiNode} steht für eine $\phi$-Instruktion aus dem IR. Sie besitzt eine
Referenz als Definition und 2 bis $n$ Uses.

\subsection{Label}

Das \textit{Label} entspricht einer Markierung, mit der Knoten in einem 
Datenflussgraphen versehen werden können. Dabei steht ein Label (oder 
\texttt{TypeLabel}, wie der Datentyp im System heißt) für einen Optimierten Typ.
Die Semantik hinter einem markierten Knoten ist, dass diese Referenz bzw. Instruktion
durch den entsprechenden Optimierten Typ ersetzt werden kann.

Es kann nicht für alle \texttt{InstructionNode}s ein Label gesetzt werden. Genauer
gesagt lassen sich ausschließlich für die Knotentypen \texttt{MethodCallNode}, 
\texttt{NewNode} und \texttt{PhiNode} ein Label setzen, da sich nur diese 
Instruktionen in einen optimierten Typ umwandeln lassen.

Das \texttt{TypeLabel} beinhaltet alle Regeln, die für die Verwendung eines
Optimierten Typen existieren. Dazu gehören

\begin{itemize}
	\item der Originale, sowie der Optimierte Typ
	\item die Methoden für die Optimierungen im optimierten Typ angeboten werden
	\item alle Methoden die darüber hinaus vom Optimierten Typ unterstützt werden
	\item Methoden, die den optimierten Typ als Rückgabewert zurückgeben
	\item kompatible Label
\end{itemize}

Dabei ist diese Liste bereits eine Abstraktion zu den Methoden, die das Interface 
besitzt. Im System lassen sich Label Definition auf 2 Arten erstellen:

\begin{enumerate} 
	\item Durch das Implementieren des Interfaces \texttt{TypeLabel}
	\item Durch das Erstellen einer \texttt{.type} Datei
\end{enumerate}

Zwar unterstützt das Kommandozeilen Tool zur Zeit nur die zweite Variante,
programmatisch lässt sich allerdings auch die erste Alternative umsetzen. Im 
Folgenden sollen die beiden Möglichkeiten zur Definition eines \texttt{TypeLabel}s
betrachtet werden.

\subsubsection{Das TypeLabel Interface}

Das Interface beinhaltet alle Methoden, die der Analyse- und Transformationsprozess
benötigt. In diesem Kapitel sollen zunächst nur die Methode betrachtet werden, die
für den Analyse Algorithmus verwendet werden, die Übrigen werden im Abschnitt 
\ref{ssec:infoLabel} betrachtet. 

\begin{description}
	\item[\texttt{canBeUsedAsReceiverFor(MethodReference)}] Legt fest, ob eine 
	markierte Referenz als Empfänger für den übergebenen Methodenaufruf dienen kann.
	\item[\texttt{canBeUsedAsParamFor(MethodReference,int)}] Bestimmt, ob eine 
	markierte Referenz als Parameter in dem gegebenen Methodenaufruf an der 
	entsprechenden Stelle (der \texttt{int} Parameter) verwendet werden kann.
	\item[\texttt{canBeDefinedAsResultOf(MethodReference)}] Sagt aus, ob die 
	gegebene Methode einen optimierten Typ zurückgeben kann. Dies impliziert, dass
	der Methodenaufruf selber auch markiert ist.
	\item[\texttt{findTypeUses(AnalyzedMethod)}] Gibt eine Menge an \texttt{Reference}
	Objekten zurück, auf denen innerhalb der gegebenen Methode eine der von 
	der Optimierung betroffenen Methode aufgerufen wird. Für diesen Algorithmus 
	existiert bereits eine Implementierung in der Klasse \texttt{BaseTypeLabel}.
	\item[\texttt{compatibleWith(TypeLabel)}] Gibt an, ob das übergebene Label 
	kompatibel mit diesem Label ist.
\end{description}

Alle diese Methoden werden von den \texttt{InstructionNode} Implementierungen 
verwendet. Wie genau das passiert wird im Abschnitt \textit{Algorithmus} beschrieben.

\subsubsection{Das .type Dateiformat}

Da das Implementieren des Interfaces eher komplex ist, wurde für die einfachere 
Definition eines Types ein Datei Format entwickelt, welches von der Komplexität des
Interfaces abstrahieren soll. In dieser werden nicht die Regeln selbst, sondern 
die Fakten beschrieben, aus denen die Regeln für den Algorithmus hergeleitet werden 
können, beschrieben.

Aus einer Datei im \texttt{type} Format wird mittels eines internen Parsers ein
\texttt{TypeLabelConfig} Objekt erzeugt, welches als \texttt{TypeLabel} Objekt für
den Algorithmus fungiert.

Für die inhaltliche Struktur der Datei wurde JSON (JavaScript Object Notation) 
gewählt eine Darstellung anzubieten, die sowohl für Menschen als auch für das 
Programm leicht zu lesen und zu verstehen ist. Die Attribute innerhalb der Datei 
werden im Folgenden beschrieben:

\begin{description}
	\item[name] Der Name des Labels
	\item[originalType] Der voll qualifizierte Name des zu ersetzenden Typs
	\item[optimizedType] Der voll qualifizierte Name des zu optimierten Typs
	\item[methodDefs] Liste von Methoden, diesen wird eine ID vergeben um sie im 
	folgenden über diese ID zu referenzieren. Ein Eintrag in dieser Liste setzt
	sich zusammen aus:
	\begin{description}
		\item[id] eine eindeutige ID für die diese Methode
		\item[desc] die Beschreibung dieser Methode. Dies ist ein eigenes Objekt 
		und besteht aus den Attributen:
		\begin{description}
			\item[name] der Name der Methode
			\item[signature] der Signatur der Methode. Zusammengesetzt aus der
			Parameterliste und der Rückgabewert. Die Typen müssen dabei in der 
			internen JVM Form angegeben werden. (Beispiel: "\texttt{(I)Ljava/lang/String;}"
			, ein Parameter vom Typ \texttt{int} und Rückgabewert vom Typ
			\texttt{java.lang.String})
		\end{description}	
	\end{description}
	\item[effectedMethods] Liste von Methoden IDs. Für diese Methoden existieren 
	optimierte Varianten in dem optimierten Typen.
	\item[supportedMethods] Liste von Methoden IDs. Diese Methoden werden auch vom
	optimierten Typ unterstützt. Es handelt sich bei diesen aber nicht um Optimierungen.
	\item[producingMethods] Liste von Methoden IDs. Alle diesen Methoden erzeugen 
	in ihrer optimierten Variante optimierte Typen.
	\item[compatibleLabel] Liste von Strings. Alle Labels die mit diesem Label
	kompatibel sind.
	\item[parameterUsage] Ein Objekt. Dabei ist jeder key die ID einer Methode 
	und der entsprechende value eine Liste von Ganzzahlen. Ein Eintrag bedeutet, 
	dass diese Methode mit einem Optimierten Typ als Parameter mit diesem Index 
	umgehen kann. 
	\item[staticFactory] Ein String. Der Name einer statischen Factory Methode mit 
	einem Übergabeparameter vom Typ des Originalen Typs. Diese muss einen entsprechenden
	Optimierten Typ zurückgeben.
	\item[toOriginalType] Ein String, Der Name einer Methode ohne Parameter, die
	aus dem optimierten Objekt, ein entsprechendes vom Originalen Typ zurückgibt.

\end{description}

\section{Algorithmus}

In diesem Abschnitt 

\subsection{Die "Bubble"}
\subsection{Umsetzung des Algorithmus}
\subsection{Umgang mit Phi-Knoten}
\chapter{Transformation}
\label{ch:trans}

In diesem Kapitel sollen die Überlegungen und der Prozess der Bytecode Transformation,
auf Basis der Resultate aus dem Analyse Prozess, vorgestellt werden. Es wird zunächst 
auf die Beschaffung der nötigen Informationen eingegangen, um im Anschluss die Regeln,
nach denen Bytecode generiert oder manipuliert wird, erläutert.

Ziel der Transformation ist es zum Einen an den Grenzen der Bubble Konvertierungen zwischen den
Originalen und den Optimalen Typen in den Sourcecode einzufügen. Zum Anderen müssen Uses
markierte Uses in entsprechende optimierte Versionen umgewandelt werden.  

\section{Lokale Variablen}
\label{sec:locals}

\subsection{Optimierte Variablen}

Um originale lokale Variablen im Bytecode nicht mit den optimierten Versionen zu 
überschreiben wurde eine Abbildung geschaffen, die jeder lokalen Variable ein Tupel 
zuweist. $l \rightarrow (L,l')$, wobei $l$ die originale lokale, $L$ ein Label und 
$l'$ die optimierte Variable für das Label $L$ darstellt. So ist sichergestellt, dass
optimierte und die entsprechende originale Referenz in zwei verschiedenen Lokalen 
geführt werden. Darüber hinaus ist diese Trennung wichtig, da die JVM die Plätze für
lokale Variablen typisiert und daher nicht an verschiedenen Stellen im Programm 
verschiedene Typen in derselben lokalen Variable gespeichert werden können.

\subsection{Variablen zu Value Numbers}

Da der IR mit den beinhalteten value numbers eine Abstraktion des eigentlichen Bytecodes
darstellt, fehlt auch jeglicher Bezug zu den eigentlichen lokalen Variablen, die von
einer spezifischen value number dargestellt wird. Darüber hinaus muss für Definition
einer Instruktion das \texttt{STORE} (schreibt die auf dem Stack liegende Referenz in 
die gegebene lokale Variable) und für alle Uses entsprechende \texttt{LOAD}s (ließt die 
gegebene lokale Variable) im Bytecode lokalisiert werden. Diese Informationen sind nötig,
da im Falle von Optimierungen, die für die entsprechende Instruktion erzeugt werden, 
die optimierten statt die originalen Referenzen geladen werden müssen.

Diese Informationen werden in der \texttt{InstructionNode} gehalten. Beim erzeugen 
eines solchen Objekts wird in der \texttt{InstructionNodeFactory} zum einen versucht
die lokalen zu den verwendeten value numbers zu erschließen und zum anderen die auf 
Position im Bytecode zu schließen an der die entsprechenden Werte auf den Stack gelegt 
werden.

Der IR, der aus einer class-Datei erzeugt wird besitzt ein privates Feld \texttt{localMap}
vom Typ \texttt{com.ibm.wala.ssa.IR.SSA2LocalMap}, welche in ihrer einzigen Implementierung
(der \texttt{com.ibm.wala.ssa.SSABuilder.SSA2LocalMap}) eine private Methode mit 
Signatur \texttt{int[] findLocalsForValueNumber(int, int)}, welche für eine gegebenen 
Bytecode Index und value number alle möglichen lokalen Variablen für diese value number 
an der gegebenen Stelle zurückgibt. Um diese Methode trotz aller Zugriffsbeschränkungen 
aufzurufen wurde eine Methode in \textit{Groovy} geschrieben um auf diese Methode 
zuzugreifen. Beim Suchen nach lokalen Variablen muss zwischen value numbers als Definition 
und als Use unterschieden werden. Das folgende Beispiel beschreibt das Problem bei 
Definitionen:

\begin{figure}[H]
	\begin{lstlisting}[language=Bytecode]
		INVOKEVIRTUAL org/example/SomeType.f()I // index 1
		ISTORE 5								// index 2
	\end{lstlisting}
	\caption{Lokale Variable für Definition}
\end{figure}

Die lokale Variable der Definition der \texttt{INVOKEVIRTUAL} Instruktion ist zum
Zeitpunkt des Methodenaufrufs noch nicht bekannt. Erst im Index 2 wird dieser Wert der 
lokalen Variablen 5 zugewiesen.

Um nun die Stellen zu finden an denen Variablen auf den Stack gelegt oder vom Stack 
gepoppt werden wurde eine einfache Stacksimulation eingeführt, wie sie in Algorithmus
\ref{alg:stack} zu sehen ist.

\begin{algorithm}[H]
	\caption{Simulation des Stacks}\label{alg:stack}
	\begin{algorithmic}[1]
		\STATE $size \gets$ Höhe des Stacks zum Zeitpunkt der Instruktion 
		\STATE $index \gets$ Index der betroffenen Referenz innerhalb des Stacks
		\STATE $bcIndex \gets$ Bytecode Index der betroffenen Instruktion
		\WHILE {\texttt{$actBlock$.getPredNodes() $= 1$}}
			\STATE \texttt{$actBlock \gets callGraph$.getBlockFor($bcIndex$)}

			\WHILE {\texttt{$bcIndex > actBlock$.getFirstInstructionIndex()}}
				\STATE $bcIndex \gets bcIndex - 1$
				\STATE $instruction \gets instructions[bcIndex]$
				\IF{\texttt{$instruction$.getPushedCount() $ > 0$}}
					\STATE $size \gets size - 1$
					\IF {index == size}
						\RETURN $bcIndex$ 
					\ENDIF
				\ENDIF
				\STATE \texttt{$size \gets size + instruction$.getPoppedCount()}
			\ENDWHILE	
		\ENDWHILE
		\RETURN $-1$ // kein Index gefunden
	\end{algorithmic}
\end{algorithm}

Dieser Algorithmus funktioniert für Definitionen, also das Suchen von \texttt{STORE}
Instruktionen, ähnlich. Der Unterschied liegt dabei ausschließlich im Inkrementieren 
(statt Dekrementieren) der $bcIndex$ Variable und dem umgekehrten Verhalten beim 
\textit{push} bzw. \textit{pop} von Werten auf bzw. vom Stack.

Diese Informationen werden in dem entsprechenden \texttt{InstructionNode} Objekt
gespeichert. Zu diesem Zweck besitzt dieser Typ drei Abbildungen ($\mathbb{N} \rightarrow 
\mathbb{N}$), die zum Zeitpunkt der Erstellung befüllt werden:

\begin{description}
	\item [\texttt{localMap}] bildet eine value number auf eine lokale Variable ab
	\item [\texttt{loadMap}] bildet eine lokale Variable auf einen Bytecode Index ab, an 
	dem diese Variable auf den Stack geladen wurde
	\item [\texttt{storeMap}] bildet eine lokale Variable auf einen Bytecode Index ab, an
	dem \texttt{STORE} diese Referenz in die entsprechende Variable schreibt
\end{description} 

\section{Bytecode Generierung}

\subsection{Informationen des TypeLabels}
\label{ssec:infoLabel}

\subsection{Konvertierung}

\subsection{Optimierung}
\chapter{Auswertung}

In diesem Kapitel soll das entwickelte System einer Auswertung unterzogen werden.
Im ersten Abschnitt soll zunächst das das verwendete Werkzeug \textit{JMH} sowie 
die allgemeine Durchführung der Tests beschrieben werden. Im zweiten Abschnitt 
folgen dann die einzelnen Tests sowie deren Auswertungen.

Die Auswertung erfolgt von Laufzeit Benchmarks. Dabei wird die Laufzeit von 
optimierten Methoden vor und nach der Optimierung durch das System gemessen und 
diese Ergebnisse miteinander verglichen. Darüber hinaus werden diese Ergebnisse 
anhand der Ausgabe des Systems begründet.

\section{Vorraussetzungen}

\subsection{Java Microbenchmarking Harness}

Microbenchmarks sind auf der JVM schwierig zu Erstellen. Das liegt vor allem 
an der \textit{Just-In-Time} Kompilierung von Java Bytecode. Die JVM stellt 
während der Laufzeit eines Programms fest, welche Abschnitte des Bytecodes 
häufig ausgeführt werden und kompiliert diese Teile dynamisch zu Maschinen Code.
Diese Kompilierung kann auch wieder rückgängig gemacht werden, wenn der übersetzte
Abschnitt nicht mehr so häufig oder in einem anderen Kontext verwendet wird. Zusätzlich 
erschwert die nicht deterministische Garbage Collection, während der 
Laufzeit, konsistente und aussagekräftige Ergebnisse.

Um diesen Problemen zu begegnen wurde das Projekt JMH verwendet. Das \textit{Java Benchmarking Harness} 
(kurz JMH) ist ein Projekt innerhalb des OpenJDKs. Es dient der Unterstützung der 
Erstellung und Ausführung von Java Microbenchmarks. Das Framework lässt sich in 
den Maven Build integrieren und bietet eine annotationsbasierte Konfiguration 
für die in Java geschriebenen Benchmarks. Zu messende Benchmarks werden als Methoden 
in einer oder mehreren Java Klassen geschrieben, die eben jenen Code ausführen dessen Ausführungszeit 
gemessen werden soll. 

\subsection{Test Durchführungen}

Alle Messungen wurden mit dem im vorherigen Abschnitt vorgestellten JMH durchgeführt. 
Gemessen werden soll die Ausführungszeit einer Methodenausführung im originalen, sowie
im Zustand nach der Anwendung des Systems auf diese Methode. Da es nicht möglich ist,
dass dieselbe Klasse, welche die optimierte Methode bereitstellt, zweimal (die originale 
sowie die optimierte Variante) im Klassenpfad aufzuführen, müssen die Messungen
jeweils für den originalen als auch den optimierten Methoden Aufruf separat durchgeführt werden.
Diese Messungen fanden auf einem von der Universität bereit gestellten Rechner statt,
der für Benchmarks zur Verfügung stand. 

Für alle Tests wurden in einem Thread 10 Durchläufen durchgeführt. Ein Durchlauf setzt
sich dabei aus Aufwärm- sowie einer Messphasen von jeweils 20 Iterationen zusammen. 
Daher existieren für jede Messung 200 Ergebnisse. Eine Iteration misst die Anzahl von 
Methoden Aufrufen in einer Sekunde und rechnet diesen Wert in einen Zeitwert um. 

Als Test-Programme wurde zum eine eigens zum Testen geschriebene Methode verwendet, 
sowie Xalan, ein freier XSLT-Prozessor der Apache Software Foundation. Beide Programme
wurden als Eingabe für das in dieser Arbeit beschriebene System verwendet und die
Ausgabe für diese Auswertung verwendet.

Die Benchmarks befinden sich in dem Unterprojekt \textit{jmh-benchmarks} welches als
Maven Eigenschaft den Pfad zum Xalan bzw. ExampleParser jar erwartet und diese
Bibliothek in den Klassenpfad aufnimmt. So wird entsprechend der Parameters ein
Benchmark entweder für die normale oder für die optimierte Variante des Java Archivs
gebaut. 

Für die automatische Auswertung der Ergebnisse wurde ein Skript geschrieben, dass die 
folgenden Schritte jeweils für die normale, als auch für die optimierte Variante ausführt:

\begin{enumerate}
	\item stößt die Maven Bauvorgänge für die beiden JMH-Benchmark Projekte an
	\item führt die resultierenden Benchmarks aus
	\item übergibt die Ausgabedatei einem R Skript um die Boxplots zu erzeugen. 
\end{enumerate}

Alle Benchmarks wurden mittels einer Oracle JVM in der Version 7 ausgeführt. 

\section{Benchmarks}

In diesem Abschnitt sollen sowohl die Ergebnisse der Messungen beschrieben, als auch 
anhand der Ergebnisse des Systems erläutert werden.

\subsection{Example Parser}

Der \textit{Example Parser} ist ein eigens für die Auswertung geschriebene Methode.
Diese soll die Möglichkeit darstellen, dass Optimierungen mit dem in hier beschriebenen
System möglich sind. Der Code stellt sich dabei wie folgt dar:

\begin{figure}[H]
	\begin{lstlisting}[language=Java]
	public String parse(String line) {
		StringBuilder result = new StringBuilder();

		for (int i = 0; i + 1 < line.length(); i+=2) {
			result.append(line.substring(i, i + 1));
		}

		return result.toString();
	}
	\end{lstlisting} 
	\caption{Example Parser}
\end{figure}

Wie dem Code zu entnehmen ist, wird hier die Methode \texttt{substring()} in 
Verbindung mit dem \texttt{StringBuilder} verwendet. Dieser Methode ist daher
optimal für die beiden verwendeten TypeLabels, wie sie in \ref{stringLabels} 
beschrieben sind. 

Aufgerufen wurde diese Methode in einem Benchmark mit einem 112 langen String. Die 
Boxplots in Abbildung \ref{bp:exampleBench} zeigen die Dauer der Ausführung
vor und nach der Optimierung durch das System. 

\begin{figure}[H]
	\centering
	\centerline{
		\includegraphics[trim=0mm 60mm 20mm 50mm,scale=0.50]{pictures/boxplot_exampleParser.pdf}
		\includegraphics[trim=20mm 60mm 0mm 50mm,scale=0.50]{pictures/vioplot_exampleParser.pdf}
	}

	
	\begin{table}[H]
	\centering
		\begin{tabular}{|l|r|r|r|}
			\hline
		   		 	  & Mittelwert & Median & \bf{$\pm$ $0,1\%$} \\
		 	\hline
		 	\hline
		 	normal    & 864,41 & 865,10 & 2,689 \\ 
		  	optimiert & 799,14 & 797,72 & 2,295 \\
		  	\hline
		  	
		\end{tabular}
	\end{table}

	\caption{Ergebnis des Example Parser Benchmark}\label{bp:exampleBench}
\end{figure}


\subsubsection{Auswertung}

Aufgrund der optimalen Abstimmung der beiden optimierten Typen \texttt{SubstringString} und
\texttt{StringListBuilder} aufeinander, können die beiden Referenzen auch ohne irgendwelche 
Konvertierungen miteinander innerhalb der Methode koexistieren. Dies führt dazu, dass
erstellte \texttt{SubstringString} Referenzen direkt dem optimierten Stringbuilder übergeben
werden können. 

Dieser Benchmark soll zeigen, dass eine Optimierung, wenn auch nur für eine eigens dafür 
entwickelte Methode möglich ist. In den folgenden Benchmarks werden realitätsnahe Methoden für 
den Test des System verwendet.
 
\subsection{Xalan}

Zur Auswertung des Systems wurde der freie XSLT-Prozessor XALAN verwendet. Diese lag in Form 
eines Java Archivs vor, dass aus dem Quellcode der Version 2.7.2 gebaut wurde. 

Um geeignete Methoden für die Benchmarks zu finden, wurden die Anzahl der \texttt{substring(..)}
Aufrufe pro \texttt{*.java} Datei gezählt und diese Kandidaten nach absteigender Anzahl sortiert.
In diesen Kandidaten wurden diejenigen 4 Methoden als Tests verwendet, welche die meisten 
Aufrufe besaßen und auch ohne größeren Aufwand (öffentliche Methoden in wenig komplexen Objekten 
oder statisch) aus der Benchmark Methode aufrufbar waren. So viel die Wahl auf die folgenden 
Methoden:

\begin{description}
	\item [\texttt{org.apache.xalan.xsltc.runtime.BasisLibrary.checkAttribQName(java.lang.String)}]
		Prüft ob der gegebene XML-Attribut Name syntaktisch valide ist.
	\item [\texttt{org.apache.xml.utils.URI.new(java.lang.String)}]
		Erzeugt ein neues URI Objekt. 
	\item [\texttt{org.apache.xpath.objects.XNumber.str()}]
		Erstellt eine String Repräsentation des XNumber Objekts.
	\item [\texttt{javax.xml.xpath.XPath.compile(java.lang.String)}]
		Kompiliert den gegebenen XPath Ausdruck in ein auswertbares XPath-Objekt.
		
		Die eigentlich identifizierte Methode war \texttt{tokenize} des verwendeten 
		Lexers. Da dieser aber die Sichtbarkeit package-private
		besitzt wurde den kompletten Use Case, in dem dieser Lexer verwendet wird, 
		zu benutzen.
\end{description}

Als Eingaben dienten Werte, die der im Projekt enthaltenen Beispiel XML Dateien entnommen wurden 
und daher einen möglichst realen Ausführungskontext darstellen.

\subsubsection{checkAttribQName}

Diese Funktion prüft die syntaktische Validität eines XSLT-Attribut Namens. Dabei wird 
der Name anhand der ersten beiden ":" getrennt, was mit Hilfe der \texttt{substring} Methode
geschieht. 

Aufgerufen wurde diese Methode mit dem Wert "xmlns:redirect". Die folgende Abbildung zeigt 
die Ergebnisse der Messung.

\begin{figure}[H]
	\centering

	\centerline{
		\includegraphics[trim=0mm 60mm 20mm 50mm,scale=0.50]{pictures/boxplot_checkAttribQName.pdf}
		\includegraphics[trim=20mm 60mm 0mm 50mm,scale=0.50]{pictures/vioplot_checkAttribQName.pdf}
	}

	\begin{table}[H]
	\centering
		\begin{tabular}{|l|r|r|r|}
			\hline
		   		 	  & Mittelwert & Median & \bf{$\pm$ $0,1\%$} \\
		 	\hline
		 	\hline
		  	normal & 50,99 & 50,92 & 0,104 \\
		 	optimiert & 51,03 & 50,96 & 0,108 \\ 
		  	\hline
		  	
		\end{tabular}
	\end{table}

	\caption{Ergebnis des checkAttribQName Benchmark}\label{bp:checkAttrBench}
\end{figure}

\paragraph{Auswertung}

Die optimierte Referenz wird als Parameter an die Methode übergeben. Also beginnt mit diesem 
Parameter Knoten die Bubble für diese Referenz und sie wird direkt zu Beginn in eine
optimierten \texttt{SubstringString} konvertiert. Allerdings werden in den folgenden beiden 
Zeilen die auf dieser Referenz die Positionen der Doppelpunkte ermittelt. Diese Methoden
werden aber nicht von dem optimierten Typ angeboten, was dazu führt, dass an beiden dieser 
Stellen eine Rückkonvertierung zum originalen Typ stattfinden muss. 

Dieser zusätzliche Aufwand sorgt dafür, dass die Optimierung des \texttt{substring} Aufrufs
durch die Konvertierungen wieder ausgeglichen wird. 

\subsubsection{instantiateURI}

Bei dieser Methode handelt es sich um den Konstruktor der Klasse\\ \texttt{org.apache.xml.utils.URI}.
Dieser erwartet einen String, der die zu repräsentierende URI enthält. Der Konstruktor teilt
die übergebene Zeichenkette in semantische Bausteine, wie das Protokoll, die Domain oder den Pfad.
Zu diesem wird die Methode \texttt{substring} wiederholt auf den übergebenen String angewendet. 

Die Messung wurde mit dem Wert "http://xml.apache.org/xalan-j/apidocs/
javax/xml/transform/package-summary.html" durchgeführt. Die folgende Abbildung präsentiert 
die Ergebnisse dieses Benchmarks.

\begin{figure}[H]
	\centering
	
	\centerline{
		\includegraphics[trim=0mm 60mm 20mm 50mm,scale=0.50]{pictures/boxplot_instantiateURI.pdf}
		\includegraphics[trim=20mm 60mm 0mm 50mm,scale=0.50]{pictures/vioplot_instantiateURI.pdf}
	}

	\begin{table}[H]
	\centering
		\begin{tabular}{|l|r|r|r|}
			\hline
		   		 	  & Mittelwert & Median & \bf{$\pm$ $0,1\%$} \\
		 	\hline
		 	\hline
		  	normal 	  & 498,79 & 498,30 & 0,965 \\
		 	optimiert & 498,22 & 497,24 & 1.086 \\ 
		  	\hline
		  	
		\end{tabular}
	\end{table}

	\caption{Ergebnis des instantiateURI Benchmarks}\label{bp:instURIBench}
\end{figure}

\paragraph{Auswertung}

Bei der Optimierung dieser Methode stellen sich einige Probleme dar. Die zur 
Optimierung verwendete Referenz wird durch einen \texttt{trim} Aufruf auf der
übergebenen URI definiert. Diese Referenz wird innerhalb der Methode häufig benutzt
und mit jedem \texttt{substring} Aufruf auf diese Referenz wird eben diese wieder 
überschrieben. Zu den sonstigen aufgerufenen Methoden auf diesen Referenzen gehören:

\begin{itemize}
  	\item \texttt{startsWith}
  	\item \texttt{length}
  	\item \texttt{charAt} 
\end{itemize}  

Wobei \texttt{length} und \texttt{charAt} auch vom Typ \texttt{SubstringString}
implementiert werden. Darüber hinaus werden diese \texttt{SubstringString} Referenzen
an die folgenden privaten Methoden übergeben. 

\begin{itemize}
 	\item \texttt{initializeScheme(String)}
 	\item \texttt{initializeAuthority(String)}
 	\item \texttt{initializePath(String)}
\end{itemize} 

In diesen Methoden werden wiederum unter anderem \texttt{substring} Aufrufe auf 
den übergebenen String ausgeführt.

All diese anderweitigen Verwendungen der optimierten Referenz sorgen dafür, dass die
Bubble an diesen Stellen endet und eine Konvertierung zum originalen Typ 
\texttt{java.lang.String} stattfindet. Diese Konvertierungen haben allerdings negative 
Auswirkungen auf die Laufzeit, wodurch die Optimierungseffekte wieder aufgehoben werden.    

\subsubsection{xNumberToString}

Eine \texttt{org.apache.xpath.objects.XNumber} repräsentiert eine Zahl innerhalb eines
XPath Ausdrucks. Die Methode \texttt{str():java.lang.String} wandelt die intern Verwaltete
Dezimal Zahl in eine String Repräsentation dieses Wertes um. Dabei wird das Ergebnis
des Ausdrucks \texttt{Double.toString(val)}, wobei \texttt{val} der aktuelle Wert des
Objektes ist, einem String zu gewiesen. Dieser String wird daraufhin in ein alternatives
Dezimal Zahlen Format umgewandelt: 

\begin{itemize}
 	\item Es werden 'NaN' bzw. 'Infinity' Strings erzeugt
 	\item Bei Ganzzahlen wird das folgende '.0' abgeschnitten
 	\item Für Zahlen mit Exponenten werden werden diese ausgeschrieben
\end{itemize} 

Um die verschiedenen Teile der Verarbeitung innerhalb der Methode mit zu berücksichtigen 
wurde der Test mit verschiedenen Eingabe durchgeführt:

\begin{itemize}
	\item Einer positiven Dezimalzahl (12,34)
	\item Einer negativen Dezimalzahl (-12,34)
	\item Einer positiven Ganzzahl (12)
	\item Einer Zahl mit negativem Exponent (0,12e-5)
\end{itemize}

Im folgenden sind die einzelnen Testergebnisse aufgeführt. Eine Auswertung dieser
Ergebnisse folgt auf diese Darstellungen.


\begin{figure}[H]
	\centering

	\centerline{
		\includegraphics[trim=0mm 60mm 20mm 50mm,scale=0.50]{pictures/boxplot_xNumberToStringPositive.pdf}
		\includegraphics[trim=20mm 60mm 0mm 50mm,scale=0.50]{pictures/vioplot_xNumberToStringPositive.pdf}
	}
	\begin{table}[H]
	\centering
		\begin{tabular}{|l|r|r|r|}
			\hline
		   		 	  & Mittelwert & Median & \bf{$\pm$ $0,1\%$} \\
		 	\hline
		 	\hline
		  	normal 	  & 121,38 & 121,17 & 0,425 \\
		 	optimiert & 162,93 & 162,97 & 0,518 \\ 
		  	\hline
		  	
		\end{tabular}
	\end{table}

	\caption{Ergebnis des xNumberToString Benchmarks (positive Dezimalzahl)}\label{bp:instURIBench}
\end{figure}



\begin{figure}[H]
	\centering

	\centerline{
		\includegraphics[trim=0mm 60mm 20mm 50mm,scale=0.50]{pictures/boxplot_xNumberToStringNegative.pdf}
		\includegraphics[trim=20mm 60mm 0mm 50mm,scale=0.50]{pictures/vioplot_xNumberToStringNegative.pdf}
	}
	\begin{table}[H]
	\centering
		\begin{tabular}{|l|r|r|r|}
			\hline
		   		 	  & Mittelwert & Median & \bf{$\pm$ $0,1\%$} \\
		 	\hline
		 	\hline
		  	normal 	  & 122,78 & 122,54 & 0,359 \\
		 	optimiert & 162,91 & 162,97 & 0,572 \\ 
		  	\hline
		  	
		\end{tabular}
	\end{table}

	\caption{Ergebnis des xNumberToString Benchmarks (negative Dezimalzahl)}\label{bp:instURIBench}
\end{figure}


\begin{figure}[H]
	\centering

	\centerline{
		\includegraphics[trim=0mm 60mm 20mm 50mm,scale=0.50]{pictures/boxplot_xNumberToStringInteger.pdf}
		\includegraphics[trim=20mm 60mm 0mm 50mm,scale=0.50]{pictures/vioplot_xNumberToStringInteger.pdf}
	}

	\begin{table}[H]
	\centering
		\begin{tabular}{|l|r|r|r|}
			\hline
		   		 	  & Mittelwert & Median & \bf{$\pm$ $0,1\%$} \\
		 	\hline
		 	\hline
		  	normal 	  & 66,69 & 66,91 & 0,488 \\
		 	optimiert & 98,42 & 97,84 & 0,605 \\ 
		  	\hline
		  	
		\end{tabular}
	\end{table}

	\caption{Ergebnis des xNumberToString Benchmarks (positive Ganzzahl)}\label{bp:instURIBench}
\end{figure}


\begin{figure}[H]
	\centering

	\centerline{
		\includegraphics[trim=0mm 60mm 20mm 50mm,scale=0.50]{pictures/boxplot_xNumberToStringExponent.pdf}
		\includegraphics[trim=20mm 60mm 0mm 50mm,scale=0.50]{pictures/vioplot_xNumberToStringExponent.pdf}
	}

	\begin{table}[H]
	\centering
		\begin{tabular}{|l|r|r|r|}
			\hline
		   		 	  & Mittelwert & Median & \bf{$\pm$ $0,1\%$} \\
		 	\hline
		 	\hline
		  	normal 	  & 569,85 & 568,86 & 1.530 \\
		 	optimiert & 605,16 & 604,93 & 1,291 \\ 
		  	\hline
		  	
		\end{tabular}
	\end{table}

	\caption{Ergebnis des xNumberToString Benchmarks (positive Dezimalzahl)}\label{bp:instURIBench}
\end{figure}

\paragraph{Auswertung}

Die Implementierung dieser Methode erzeugt innerhalb der Methode eine String Repräsentation des
im Objekt verwalteten Gleitkommazahl. Dieser Wert ist vom Typ \texttt{double}. Der 
String wird über einen Aufruf von \texttt{Double.toString(double)} erzeugt, dem der 
\texttt{double} Wert der Klassen Eigenschaft übergeben wird. Diese String Referenz wird innerhalb der 
Methode formatiert. Dies geschieht über Kontrollflüsse die abhängig von der ursprünglichen Zahl sind.
Die vier Vorgestellten Benchmarks decken die verschiedenen Verarbeitungsschritte ab. Die auf der
Referenz in jedem Fall aufgerufenen Methoden (abgesehen von \texttt{substring}) sind:

\begin{itemize}
	\item \texttt{length()}
	\item \texttt{charAt(int)}
	\item \texttt{indexOf(char)}
\end{itemize}

Wobei die Methode \texttt{indexOf(char)} nicht von dem optimierten Typen unterstützt wird.
Da dieser Aufruf allerdings für jede Zahl $\neq 0$ aufgerufen wird (es wird nach dem '\texttt{e}' in
Exponenten gesucht), muss vor diesem Aufruf jeder \texttt{SubstringString} in einen String
konvertiert werden. 

Diese Konvertierungen für jeden Aufruf, zusätzlich zu denen hin zu einem \texttt{SubstringString}
und zurück zu einem \texttt{String} vor dem Zurückgeben des Strings aus der Methode, führen zu 
einer beträchtlichen Anstieg der Laufzeit in der optimierten Variante dieser Methode.


\subsubsection{compileXPath}

\chapter{Fazit}

\end{document}